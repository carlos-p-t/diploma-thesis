\subsection{Data for Object Detection Task}\label{sec:datadetection}

The Mapillary Vistas Dataset contains a total of 25,000 high-resolution images. Due to GPU and storage constraints, a portion of the original dataset was used to train the object detector. The method of selecting images to create the subset is described in further subsections. The data is organized into training, validation, and testing subsets. Table 2.1 shows the details of the distributions of these subsets..\\

\begin{table}[h!]
\centering
\renewcommand{\arraystretch}{1.3} % Adjust row height
\begin{tabular}{|c|c|c|}
\hline
\textbf{} & \textbf{Mapillary Vistas (Original)} & \textbf{Mapillary Vistas Subset} \\ \hline
\textbf{Training}    & 18000  & 6000  \\ \hline
\textbf{Validation}  & 2000   & 1200  \\ \hline
\textbf{Testing}     & 5000   & 2500  \\ \hline
\textbf{Total}       & 25000  & 9700  \\ \hline
\end{tabular}
\caption{Comparison of the full Mapillary Vistas dataset and its subset used for training our object detector.}
\label{tab:mapillarydata}
\end{table}

After training the object detector with the Mapillary Vistas subset, the BillboardLamac Dataset was used to fine-tune the results. This dataset contains a specific subset of images exclusively for object detection. The details of this subset can be found in Table 2.\\

\begin{table}[h!]
\centering
\renewcommand{\arraystretch}{1.3} % Adjust row height
\begin{tabular}{|c|c|}
\hline
\textbf{} & \textbf{BillboardLamac Dataset} \\ \hline
\textbf{Training}    & 795 \\ \hline
\textbf{Validation}  & 199  \\ \hline
\textbf{Testing}     & 219  \\ \hline
\textbf{Total}       & 1213 \\ \hline
\end{tabular}
\caption{BillboardLamac Dataset subsets for fine-tuning the object detector.}
\label{tab:lamacdata}
\end{table}